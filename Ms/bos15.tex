\documentclass[12pt]{article}
% \usepackage[breaklinks=true]{hyperref}
% \usepackage[html,png]{tex4ht}
\usepackage{color}
\usepackage{amsmath,amssymb,amsthm}
\usepackage{natbib}
\usepackage{array}
\usepackage{booktabs, multicol, multirow}
\usepackage[nohead]{geometry}
\usepackage[singlespacing]{setspace}
\usepackage[bottom]{footmisc}
\usepackage{floatrow}
\usepackage{float}
\usepackage{caption}
\usepackage{indentfirst}
\usepackage{lscape}
\usepackage{floatrow}
\usepackage{epsfig}
\usepackage[usenames,dvipsnames,svgnames,table]{xcolor}
\usepackage[colorlinks=true,
            urlcolor=RawSienna,
            linkcolor=RawSienna,
            citecolor=NavyBlue]{hyperref}
\floatsetup[table]{capposition=top}
\floatsetup[figure]{capposition=top}


\newcommand{\beq}{\begin{equation}}
\newcommand{\eeq}{\end{equation}}

\newcommand{\cD}{{\mathcal D}}
\newcommand{\cF}{{\mathcal F}}
\newcommand{\todo}[1]{{\color{red}{TO DO: \sc #1}}}
\renewcommand{\baselinestretch}{1.5}

\title{Student Evaluations of Teaching (Mostly) Do Not Measure Teaching Effectiveness}
\author{Anne Boring, Kellie Ottoboni, Philip B.~Stark}
\date{Draft \today}
\begin{document}
\maketitle

\newpage
\begin{quotation}
    \emph{The truth will set you free, but first it will piss you off.}
    
     \hfill Gloria Steinem

\begin{abstract}

We test whether student evaluations of teaching (SET) 
primarily measure teaching effectiveness, using 
two datasets:
23,001~SET of 379~instructors by 4,423~students in six 
mandatory first year courses in a five-year natural experiment at a French university, 
and
43~SET for 4~sections of an online course in a randomized, controlled, 
blind experiment at a US university.
We study relationships among SET and the genders of students and actual and
perceived genders of instructors, grade expectations, and grades.
Nonparametric tests show that on average students rate apparently female instructors lower 
by an amount that is 
large and statistically significant, even on putatively objective measures such as promptness.
On average, students who expect higher grades give higher SET, 
by an amount that is statistically significant.
However, neither SET nor apparent instructor gender is significantly associated 
with student performance on anonymously graded uniform final exams. 
SET are more sensitive to student biases and grade expectations than to teaching 
effectiveness.
The bias varies by student and by subject, so it is effectively impossible to adjust for. 
Relying on SET for personnel decisions disadvantages female instructors.


% Nonparametric permutation tests that aggregate within 1,194 course sections show: 
%\begin{itemize}
%   \item the association between SET and final 
%            exam scores is negative but insignificant
%            ($P \approx 0.70$)
%   \item the association between SET and grade 
%            expectations is positive and highly significant
 %           ($P \approx 0.00$)
%   \item the association between instructor gender 
%            and final exam scores is insignificant
%            (students of male instructors do worse, $P\approx 0.51$ overall, $0.76$ for male students, $0.65$ for female students)
%   \item the association between instructor gender and SET is highly significant---because male
%             students rate male instructors higher 
%            (men get higher ratings, $P \approx 0.00$ overall, $0.00$ for male students,
%            $0.49$ for female students)
%\end{itemize}
%These relationships vary by discipline. 

%all first-year students take the same courses 
%(economics, history, political science, sociology, and %political institutions). 
%Students are assigned to sections of those courses as if %at random, creating a natural experiment.
%Final exams are set for the entire course
%by the professor rather than the section instructor, and %are graded anonymously.
%Hence, final exam scores are a proxy for the %effectiveness of the section instructors.
%SET are mandatory.
%
%Student responses fail simple tests of data quality.
%For instance, 29\% of students report spending impossible amounts of time
%on their courses.

\end{abstract}

\newpage

\end{quotation}

\section{Background}
Student evaluations of teaching (SET) are used widely 
in decisions about hiring, promoting, and firing instructors. 
Measuring teaching effectiveness is difficult---for students,
faculty, and administrators alike.
Universities generally treat SET as if they primarily
measure teaching effectiveness or teaching quality.
This assumption may not be justified.
SET may be more sensitive to something other than teaching effectiveness:
they may be strongly \emph{biased}.\footnote{%
  \citet[p.17]{Centra2000} define bias to occur when ``a teacher
   or course characteristic affects teacher evaluations, either positively or
negatively, but is unrelated to criteria of good teaching, such as increased student learning.'' 
}

Randomized experiments \citep{Carrell2010a,Braga2014}
have shown that students confuse grades 
and grade expectations with the long-term value of a course and that SET are not associated with 
student performance in follow-on courses, i.e., with teaching effectiveness. 
On the whole, high SET amount to a reward students give 
instructors who are easy graders or who ``teach to the test.''  

Gender matters too.
\citet{Boring2015} finds that SET are affected by gender biases and stereotypes. 
Male first-year undergraduate students give more \textit{excellent} scores to male instructors,
even though there is no difference between
the academic performance of male students of male and of female instructors.
Experimental work by \citet{MacNell2014} finds that when students think
an instructor is female,
students rate the instructor lower on every aspect of teaching,
including putatively objective measures such as timeliness.

Here, we apply nonparametric permutation tests to data from \citet{Boring2015} and 
\citet{MacNell2014} to investigate whether SET primarily measure teaching effectiveness or 
biases using a higher level of statistical rigor.
The two main sources of bias we study are students' grade expectations and the gender of the 
instructor. 
We also investigate variations in bias by discipline and by student gender.

Permutation tests allow us to avoid 
contrived, counterfactual assumptions about
parametric generative models for the data, which regression-based methods (including
ordinary linear regression, mixed effects models, logistic regression, etc.) and
methods such as $t$-tests and ANOVA generally require.
The null hypotheses for our tests are that some 
characteristic---e.g., instructor gender---amounts to an arbitrary label and might as well
have been assigned at random. 

We work with course-level summaries to match how institutions use SET: 
typically, SET for a course
are averaged, and those averages are compared across instances of the course,
across courses in a department, across instructors, and across departments.
\citet{StarkFreishtat2014} discuss statistical problems with this reduction to 
and reliance upon averages.

We find that the association between SET and an objective measure of teaching effectiveness 
is weak and not statistically significant.
In contrast, the association between SET and (perceived) instructor gender 
is large and statistically significant:
instructors who students believe are male receive significantly higher average SET.
In the French data, \emph{male} students tend to rate male instructors higher
than they rate female instructors, with little difference in ratings by female students;
in the US data, \emph{female} students tend to rate (perceived) male instructors 
higher than they rate (perceived) female instructors, with little difference in ratings by male students. 
The French data also show that gender biases vary by course topic, and 
that grade expectations and SET are strongly associated.

We therefore conclude that SET primarily do not measure teaching effectiveness; that 
they are strongly and non-uniformly biased by factors such as the gender of the instructor
and student; and that it is impossible to adjust for these biases. 
SET should not be relied upon as a measure of teaching effectiveness.
Relying on SET for personnel decisions has disparate impact by gender, in general. 

\section{Data}
\subsection{French Natural Experiment}
The data, collected between 2008 and 2013, are a census of 23,001 SET from
4,423 first-year students at a French university students (57\% women) in 1,177
sections, taught by 379 instructors (34\% women). 
The data are not public, owing to French restrictions on human subjects data.
\citet{Boring2015} describes the data in detail.
Key features include:
\begin{itemize}
   \item All first-year students take the same six mandatory courses: 
            History, Macroeconomics, Microeconomics, 
            Political Institutions, Political Science, and Sociology.
            Each course has one (male) professor
            who delivers the lectures, to groups of approximately 900 students. 
            Courses have sections of 10--24 students. 
            Those sections are taught by a variety of instructors, male and female.
            The instructors have considerable pedagogical freedom.
    
   \item Students enroll in ``triads'' of sections of these courses (three courses per semester). 
            The enrollment process does not allow students to select individual instructors.
            The assignment of instructors to sets of students is as if at random,
            forming a \emph{natural experiment}.
            The assignment can be treated as if it is independent across 
            courses.
            
   \item Section instructors assign interim grades during the semester.
            Interim grades are known to the students before they submit SET.
            Interim grades are a reasonable proxy for student grade expectations.
            
   \item Final exams are written by the professor, not the section instructors.
            Students in all sections of a course in a given year take the same final.
            Final exams are graded anonymously, except in Political
            Institutions, which we therefore omit from analyses involving final exam scores.
            To the extent that the final measures appropriate learning outcomes, 
            performance on the final exam is a measure of the effectiveness of
            an instructor: in a given course in a given year,
            students of more effective instructors should do better on
            the final exam, on average, than students of less effective instructors.
    
   \item SET are mandatory: response rates are nearly 100\%.
   
\end{itemize}

SET include closed-ended and open-ended questions.
The item that attracts the most attention is the \emph{overall score}, 
which is treated as a summary of the other items.
The SET data include students' individual evaluations of section
instructors in microeconomics, history, political institutions, and 
macroeconomics for the five academic years 2008--2013, and for 
sociology and political science for the three academic years 2010--2013 
(these two subjects were introduced in 2010). 
The SET are anonymous to the instructors, who have access to the ratings only after 
all grades have been officially recorded.  

\begin{table}[htbp]
  \centering
  \footnotesize 
  \caption{Summary statistics of sections}
    \begin{tabular}{lccc}
    \toprule 
    course     & \# sections & \# instructors  & \% Female instructors  \\
   \midrule
  \textbf{Overall} &  \textbf{1,194} & \textbf{379}  &\textbf{33.8\%} \\
    History    &               230 &      72          &   30.6\% \\
    Political Institutions  &  229 &      65          &   20.0\% \\    
    Microeconomics   &         230 &      96          &   38.5\% \\
    Macroeconomics   &         230 &      93          &   34.4\% \\
    Political Science &       137 &      49          &   32.7\% \\
    Sociology   &              138 &      56          &   46.4\%    \\
    \bottomrule
    \end{tabular}%
 \label{tab:description}%
 
\textit{Data for a section of Political Institutions that 
had an experimental online format are omitted.
Political Science and Sociology originally were not in the triad system; 
students were randomly assigned by the administration to different sections.
} 

\end{table}%
\normalsize
%Overall, 34\% of the 1,194 sections were taught by women 
%(Table~\ref{tab:description}), but the percentage varies by discipline. 
%Only 20\% of Political Institutions sections were taught by women. 
%Sociology is divided approximately equally between male and female instructors 
%(46.4\% of sections were taught by women). 
%Microeconomics and Macroeconomics had more instructors total, because of higher turnover.

\subsection{U.S. Randomized Experiment}
These data, described in detail by \cite{MacNell2014}, are available at 
\url{http://n2t.net/ark:/b6078/d1mw2k}.
Students in an online course were randomized into six sections of about a dozen students each, 
two taught by the primary professor,
two taught by a female graduate teaching assistant (TA), and two taught by a male TA.
In one of the two sections taught by each TA, the TA used her or his
true name; in the other, she or he used the other TA's identity.
Thus, in two sections, the students were led to believe they were being taught by a woman
and in two they were led to believe they were being taught by a man.
Students had no direct contact with TAs: the primary interactions were through
discussion boards.
The TA credentials presented to the students were comparable; the TAs covered
the same material; and assignments were returned at the same time in all sections.

SET included an overall score and questions relating to
professionalism, respectfulness, care, enthusiasm, communication, helpfulness,
feedback, promptness, consistency, fairness, responsiveness, praise, knowledge, 
and clarity.
Fourty-seven students in the four sections taught by TAs finished the class,
of whom 43 completed SET.
The SET data include the genders and birth years of the students; the grade data do not.
The SET data are not linked to the grade data.

\section{Methods} \label{sec:methods}
We use permutation tests based on the as-if-random (French natural experiment)
or truly random (US experiment) assignment of students
to class sections.
In most cases, our tests are stratified: we randomize separately in different
groups of students, independently across the groups.
For instance, for the US data the randomization is stratified on the actual TA:
students are randomized within the two sections taught by each TA,
but students assigned to different TAs comprise different strata.
For the French data, the randomization is stratified on course and year:
students in different courses or in different years comprise different strata.
The null distributions of the test statistics\footnote{%
 The test statistics are correlations of variables or differences in means across experimental
 conditions, aggregated across strata.
}
are induced by this random
assignment, with no assumption about the distribution of SET or other variables, 
no parameter estimates, and no models.

\subsection{Illustration: French natural Experiment} \label{sec:boringMethods}
The selection of course sections by students at the French university---and the implicit 
assignment of instructors to sets of students---is as if at random within sections of each
course each year, independent across courses and across years.
The university's ``triad'' system groups students in their classes across disciplines,
building small cohorts for the year.
Hence, the randomization for our test keeps these groups of students intact.
Stratifying on course topic and year keeps students who took the same final
exam grouped in the randomization and honors the design of the natural experiment.

Teaching effectiveness is multidimensional \citep{Marsh1997} and difficult to define,
much less measure. 
But whatever it is, effective teaching should promote student learning:
students of an effective instructor should have better learning outcomes
than students of an ineffective instructor have.
In the French university, in all courses other than Political Institutions,\footnote{%
 The final exam in Political Institutions is oral and hence not graded anonymously.
}
students in every section of a course in a given year take the same ononymously graded
final exam.
To the extent that final exams are designed well, scores on these finals reflect relevant 
learning outcomes for the course.
Hence, in each course each year, students of more effective instructors should do better 
on the final, on average, than students of less effective instructors.

Consider testing the hypothesis that SET are unrelated to performance on the final
against the alternative that, all else equal, students of instructors who get higher average SET
learn more.
For this hypothesis test, we omit Political Institutions.

The test statistic is the sum over courses and years of the Pearson correlation between
mean SET and mean final exam score among sections of each course each year.
If SET do measure instructors' contributions to learning, we would expect this sum of
correlations to be positive: sections with above-average mean SET in each discipline each year
would tend to be sections with above-average mean final exam scores.
How surprising is the observed sum of correlation coefficients, if there is no overall
connection between mean SET and mean final exam for sections of a course?

There are 950 ``individuals,'' course sections of subjects other
than Political Institutions.
Each of the 950 course sections has an average SET and an average final exam score.
These fall in $3\times5 + 2 \times 3 = 21$ year-by-course strata.
Under the randomization, within each stratum, instructors
are assigned sections independently across years and courses, with
the number of sections of each course that each instructor teaches each year held fixed.
For instance, if in 2008 there were $N$ sections of History taught by $K$ instructors in all,
with instructor $k$ teaching $N_k$ sections, then in the randomization,
all
\beq
    {N}\choose{N_1 \cdots N_K}
\eeq
ways of assigning $N_k$ of the $N$ 2008 History sections to instructor $k$ 
would be equally likely.
The same would hold for sections of other courses and other years;
each combination of assignments across courses and years would be equally likely:
the assignments are independent across strata.

Under the null hypothesis that SET have no relationship to final exam scores,
final exam averages for sections in each course each year are exchangeable given the average SET
for the sections.
Imagine ``shuffling'' (i.e., permuting)
the average final exam scores across sections of each course
each year, independently
for different courses and different years.
For each permutation, compute the Pearson correlation between average SET for each section
and average final exam score for each section, for each course, for each year.
Sum the resulting 21 Pearson correlations.
The probability distribution of that sum is the null distribution of the test statistic.
The $p$-value is the upper tail probability of that distribution beyond the observed value of the 
test statistic.

The randomization holds triads fixed, to allow for cohort effects
and to match the natural experiment.
Hence, the test is conditional on which students happen to sign up for which triad triads.
However, if we test at level no greater than $\alpha$ conditionally 
on the grouping of students into triads, 
the unconditional level of the resulting test across all possible groupings is no 
greater than $\alpha$:
\begin{eqnarray}
   \Pr \{ \mbox{ Type I error } \} &=& \sum_{\mbox{all possible sets of triads}} \Pr \{ \mbox{ Type I error } | 
   \mbox{ triads } \} \Pr\{\mbox{ triads } \} \nonumber \\
   &\le& 
    \sum_{\mbox{all possible sets of triads}} \alpha \Pr\{\mbox{triads } \} \nonumber \\
    &=& \alpha \sum_{\mbox{all possible sets of triads}} \Pr\{\mbox{triads } \} \nonumber \\
    &=& \alpha.
\end{eqnarray}

It is not practical to enumerate all possible permutations of sections within courses
and years, so we estimate the $p$-value by performing $10^5$ 
random permutations within each stratum, finding the value of the test statistic for each
overall assignment, and comparing the observed value of the test statistic to the 
empirical distribution of those $10^5$ random values.
The probability distribution of the number of random permutations assignments for 
which the test statistic is greater than or equal to its observed value is Binomial with 
$n$ equal to the number of overall
random permutations and $p$ equal to the true $p$-value.
Hence, the standard error of the estimated $p$-values is hence no larger than 
$(1/2)/ \sqrt{10^5}
\approx 0.0016$.
Code for all our analyses is at 
\url{https://github.com/kellieotto/SET-and-Gender-Bias}.
Results for the French data are below in section~\ref{sec:Fr-results}.

\subsection{Illustration: US Experiment}
To test whether perceived instructor gender affects SET in the US experiment,
we use the Neyman ``potential outcomes'' framework.
A fixed number $N$ of individuals---e.g., students or classes---are assigned either
randomly or as if at random by Nature into 
$k \ge 2$ groups of sizes $N_1, \ldots, N_k$.
Each group receives a different treatment.
``Treatment'' is notional. 
For instance, the treatment might be the gender of the
class instructor.

For each individual $i$, we observe a numerical response $R_i$.
If individual $i$ is assigned to treatment $j$, then $R_i = r_{ij}$.
The numbers $\{r_{ij}\}$ are considered to have been fixed before the experiment.
Implicit in this notation is the \emph{non-interference} assumption that
each individual's response depends only on the treatment that individual receives, 
and not on which treatments other individuals receive.

We observe only one potential outcome for individual $i$, 
depending on which treatment she or he receives.
In this model, the responses $\{R_i\}_{i=1}^N$ are random, but only because 
individuals are assigned to treatments at random.

In the experiment conducted by \citet{MacNell2014},
$N$ students were assigned at random to six sections of an online course,
of which four were taught by TAs.
The analysis focuses on the four sections taught by TAs.
We condition on the assignment of students to the two sections taught by the professor.
Each remaining student $i$ could be assigned to any of $k=4$ treatment conditions:
either of two TAs, each identified as either male or female.
The assignment of students to sections was made at random: each of the
\beq
 {{N}\choose{N_1 N_2 N_3 N_4}} = \frac{N!}{N_1! N_2! N_3! N_4!}
\eeq
possible assignments of $N_1$ students to TA~1 identified as male,
$N_2$ student to TA~1 identified as female, etc., was equally likely.

Let $r_{i1}$ and $r_{i2}$ be the ratings student $i$ would give TA~1 when 
TA~1 is identified as male and as female, respectively; and let 
$r_{i3}$ and $r_{i4}$ the ratings student $i$ would give TA~2 when that TA
is identified as male and as female, respectively.
Typically, the null hypotheses we test assert that for each $i$, some subset of
$\{r_{ij}\}$ are  equal.
For assessing whether the identified gender of the TA affects SET,
the null hypothesis is that for each $i$,
$r_{i1} = r_{i2}$ (the rating the $i$th student would give TA~1 is the same,
whether TA~1 is identified as male or female), 
and $r_{i3} = r_{i4}$ (the rating the $i$th student would give TA~2 is
the same, whether TA~2 is identified as male or female).
Different students might give different ratings under the same treatment condition
(the null does not assert that $r_{ij} = r_{\ell j}$ for $i \ne \ell$), and
the $i$th student might 
give different ratings to TA~1 and TA~2
(the null does not assert that $r_{i1} = r_{i3}$).
The null hypothesis makes no assertion about the population distributions of 
$\{r_{i1}\}$ and $\{r_{i3}\}$, nor does it assert that $\{r_{ij}\}$ are 
a sample from some super-population.

For student $i$, we observe exactly one of $\{r_{i1}, r_{i2}, r_{i3}, r_{i4}\}$.
If we observe $r_{i1}$, then---if the null hypothesis is true---we also know what $r_{i2}$ is,
and vice versa, but we do not know anything about $r_{i3}$ or $r_{i4}$.
Similarly, if we observe either $r_{i3}$ or $r_{i4}$ and the null hypothesis is true,
we know the value of both, but we do not know anything about $r_{i1}$ or $r_{i2}$.

Consider the average SET (for any particular item)
given by the $N_2 + N_4$ students
assigned to sections taught by an apparently female TA, minus the 
average SET given by the $N_1 + N_3$ students
assigned to sections taught by an apparently male TA.
This is what \cite{MacNell2014} tabulate as their key result.
If the perceived gender of the TA made no difference in how students rated 
the TA, we would expect the difference of averages to be close to
zero.\footnote{%
  We would expect it to be a least a little different from zero both because of the luck of the draw
  in assigning students to sections and because students might rate the two TAs
  differently, regardless of the TA's perceived gender, and the groups are not all the same size.
}
How ``surprising'' is the observed difference in averages?

Consider the
\beq
  {{N_1 + N_2} \choose {N_1}} \times {{N_3+N_4} \choose {N_3}}
\eeq
assignments that keep the same $N_1 + N_2$ students in TA~1's
sections (but might change which of those sections a student is in) 
and the same $N_3 + N_4$ students in TA~2's sections.
For each of those assignments, we know what $\{R_i\}_{i=1}^N$ would
have been if the null hypothesis is true: each would be exactly the same
as its observed value, since those
assignments keep students in sections taught by the same TA.
Hence, we can calculate the value that the test statistic would have had for each
of those assignments.

Because all ${N}\choose{N_1 N_2 N_3 N_4}$ possible assignments of students
to sections are equally likely, these 
${{N_1 + N_2} \choose {N_1}} \times {{N_3+N_4} \choose {N_3}}$ 
assignments are also equally likely.
The fraction of those assignments that produce a value of the test statistic that
is at least as large (in absolute value) as the observed value of the test statistic
is the $p$-value of the null hypothesis that students give the same rating (or none) to
an TA, regardless of the gender that TA appears to have.

This test is conditional on which of the students are assigned to each of the two 
TAs, but if we test at level no greater than $\alpha$ conditionally on the
assignment, the unconditional level of the resulting test across all assignments is no 
greater than $\alpha$, as shown above.

In principle, one could enumerate all the equally likely assignments and compute the value
of the test statistic for each, to determine the (conditional) null distribution of the test
statistic.
In practice, there are prohibitively many assignments
(for instance, there are ${{23}\choose{11}}{{24}\choose{11}} > 3.3\times 10^{12}$ 
possible assignments
of 47 students to the 4 TA-led sections that keep constant which
students are assigned to each TA).
Hence, we estimate $p$-values by simulation, drawing $10^5$ equally likely assignments
at random.
The distribution of the number of simulated assignments for which the test statistic
is greater than or equal to its observed value is Binomial with $n$ equal to the number of
simulated assignments and $p$ equal to the true $p$-value.
Hence, the standard error of the estimated $p$-values is hence no larger than 
$(1/2)/ \sqrt{10^5}
\approx 0.0016$.
Code for all our analyses is at \url{https://github.com/kellieotto/SET-and-Gender-Bias}.
Results for the US data are in section~\ref{sec:US-results}

\section{The French Natural Experiment} \label{sec:Fr-results}
In this section, we test hypotheses about relationships among
SET, teaching effectiveness, grade expectations, and student and instructor gender.
Our tests aggregate data within course sections, to match how SET are typically
used in personnel decisions. 
We use the sum of Pearson correlations across strata as the test statistic,\footnote{%
   As discussed above, we find $p$-values from the permutation distribution, not from
   the theoretical distribution of the Pearson correlation 
   under the parametric assumption of bivariate normality.
} 
which allows us to test both for 
differences in means (which can be written as correlations with a dummy variable) and for 
association with ordinal or quantitative variables.

In these analyses, individual $i$ is a section of a course; the ``treatment'' is the instructor's gender, 
the average interim grade, or the average final exam score;
and the ``response'' is the average SET or the average final exam score.
Strata consist of all sections of a single course in a single year.

Our tests for overall effects stratify on the discipline, to account for systematic
differences across departments:
the hypothetical randomization shuffles characteristics among courses in a given
department, but not across departments.
We also perform tests separately in different departments, and in some cases separately by
student gender.

\subsection{SET and final exam scores} \label{sec:Fr-set-final}
We test whether average SET scores and average final exam scores for course
sections are associated (Table~\ref{tab:finalexam}). 
The null hypothesis is that the pairing
of average final grade and average SET for sections of a course each year is as if at random,
independent across courses and across years.
We test this hypothesis overall and separately by discipline 
(Table~\ref{tab:finalexam}), using the sum of Pearson correlations across strata, as described 
in section~\ref{sec:boringMethods}.
If the null hypothesis were true, we would expect the test statistic to be 
close to zero.
On the other hand, if SET do measure teaching effectiveness, we would expect average
SET and average final exam score to be positively correlated within courses within years, 
making the test statistic positive. 

The numbers show that SET scores do not measure teaching effectiveness well, overall:
the one-sided $p$-value for the hypothesis that the correlation is zero is 0.70 (the
observed correlation is actually negative). 
Separate tests by discipline find that for microeconomics and macroeconomics, the 
association is positive and statistically
significant ($p$-values of 0.03 and 0.04). 
For history, political science, and sociology, which have exams that are graded anonymously,
the association is not significant ($p$-values 0.31, 0.53, and 0.27, respectively). 

\begin{table}[htbp]
  \centering
  \footnotesize 
  \caption{Correlation between SET and final exam scores}
    \begin{tabular}{lcc}
    \toprule 
                        & $\rho$  & $p$-value  \\
   \midrule
    Overall &            -0.02 &       0.70  \\
    History &             0.03 &       0.31  \\
    Macroeconomics &      0.12 &       0.04  \\
    Microeconomics &      0.13 &       0.03  \\
    Political science &  -0.01 &       0.53  \\
    Sociology &           0.05 &       0.27  \\
    \bottomrule
    \end{tabular}%
 \label{tab:finalexam}%
 
\textit{Note: $p$-values are one-sided, since we expect higher SET to be associated
with higher final exam scores.}
\end{table}%
\normalsize


\subsection{SET and Instructor Gender} \label{sec:Fr-set-gender}
The second null hypothesis we test is that the pairing (by section) of 
instructor gender and SET is as if at random within courses each year, independently
across years and courses.
If gender does not affect SET, we would expect the correlation between average SET
and instructor gender to be small in each course in each year.
On the other hand, if students tend to rate instructors of one gender higher, we would
expect the sum of these correlations to be large in absolute value.
We find that average SET are significantly associated with instructor gender, with male instructors
getting higher ratings (overall $p$-value 0.00). 
Male instructors get higher SET on average in every discipline  (Table~\ref{tab:instructorgender})
with two-sided $p$-values ranging from $0.07$ for history to $0.58$ for microeconomics.

\begin{table}[htbp]
  \centering
  \footnotesize 
  \caption{Association between SET and instructor gender}
    \begin{tabular}{lcc}
    \toprule 
                          & $\rho$  & $p$-value     \\
   \midrule
    Overall &                 0.10       & 0.00     \\
    History &                 0.12       & 0.07     \\
    Political institutions &  0.11       & 0.10     \\
    Macroeconomics &          0.11       & 0.08     \\
    Microeconomics &          0.04       & 0.58     \\
    Political sciences &      0.07       & 0.43     \\
    Sociology &               0.10       & 0.26     \\
    \bottomrule
    \end{tabular}%
 \label{tab:instructorgender}%
  
  \textit{Note: $p$-values are two-sided.}
\end{table}%
\normalsize

\subsection{Instructor Gender and Learning Outcomes} \label{sec:Fr-gender-final}
Do men receive higher SET scores overall because they are better instructors? 
The third null hypothesis we test is that the pairing (by course) of instructor gender and
average final exam score is as if at random within courses each year, independent
across courses and across years.
If this hypothesis is true, we would expect the sum of correlations to be small.
If the effectiveness of instructors differs systematically by gender,
we would expect sum of correlations to be large in absolute value. 
Table~\ref{tab:genderfinal} shows that on the whole, students of male instructors
perform worse on the final than students of female instructors, but by an amount that is
not statistically significant ($p$-value 0.51 overall).
The only discipline in which students of male instructors perform better is political science,
but by an amount that is not statistically significant ($p$-value 0.79)
This suggests that male instructors are not noticeably more effective than female instructors, 
and perhaps are less effective:
The statistically significant difference in SET scores for male and female instructors
does not seem to reflect a difference in their teaching effectiveness.


\begin{table}[htbp]
  \centering
  \footnotesize 
  \caption{Association between final exam scores and instructor gender}
    \begin{tabular}{lcc}
    \toprule 
                     & $\rho$  & $p$-value    \\
   \midrule
    Overall &            -0.02       & 0.51      \\
    History &            -0.06       & 0.39      \\
    Macroeconomics &      0.00       & 0.97      \\
    Microeconomics &     -0.03       & 0.63      \\
    Political science &  0.02       & 0.79      \\
    Sociology &          -0.00       & 0.97      \\
    \bottomrule
    \end{tabular}%
 \label{tab:genderfinal}%
 
  \textit{Note: $p$-values are two-sided.}
\end{table}%
\normalsize


\subsection{Gender Interactions}
Why do male instructors receive higher SET scores? 
Stratifying the analysis by student gender shows that
male students tend to give higher SET scores to male instructors 
(Table~\ref{tab:genderconcordance}). 
These permutation tests confirm the results found by \citet{Boring2015}. 
Gender concordance is a good predictor of SET scores for men ($p$-value 0.00 overall). 
Male students give significantly higher SET scores to male instructors in  
History ($p$-value 0.00), Macroeconomics ($p$-value 0.04), Political Science ($p$-value 0.06), 
Political Institutions ($p$-value 0.07), and Microeconomics ($p$-value 0.10). 
Male students give higher SET scores to male instructors in Sociology as well, but the
effect is not statistically significant ($p$-value 0.15). 

Although gender concordance predicts overall satisfaction scores for male students, 
it does not for female students ($p$-value 0.49 overall). 
The correlation is negative in some fields (History, Political Institutions, 
Macroeconomics, and Sociology) and positive in others 
(Microeconomics and Political Science), but in no case statistically significant 
($p$-values range from 0.19 to 0.97).

\begin{table}[htbp]
  \centering
  \footnotesize 
  \caption{Association between SET and gender concordance}
    \begin{tabular}{lccccc}
    \toprule 
          & \multicolumn{2}{c}{Male student}  &  & \multicolumn{2}{c}{Female student} \\
      & $\rho$  &  $p$-value &  & $\rho$  &  $p$-value    \\
   \midrule
      \quad  Overall &                 0.15       & 0.00 & &  0.02       & 0.49      \\
      \quad  History &                 0.18       & 0.00 & & -0.04       & 0.54      \\
      \quad  Political institutions &  0.12       & 0.07 & & -0.09       & 0.19       \\
      \quad  Macroeconomics &          0.14       & 0.04 & & -0.08       & 0.21     \\
      \quad  Microeconomics &          0.11       & 0.10 & &  0.03       & 0.67       \\
      \quad  Political sciences &      0.16       & 0.06 & &  0.00       & 0.97      \\
      \quad  Sociology &               0.12       & 0.15 & & -0.05       & 0.53      \\
    \bottomrule
    \end{tabular}%
 \label{tab:genderconcordance}%
  
  \textit{Note: two-sided $p$-values are reported.}
\end{table}%
\normalsize

Do male instructors receive higher SET scores from male students because their 
teaching styles match male students' learning styles? 
If so, we would expect male students of male instructors to perform better on the final exam. 
However, they do not (Table~\ref{tab:finalconcordance}). 
If anything, male students of male instructors perform worse overall on the final exam (the correlation is negative but statistically insignificant, with a $p$-value 0.76). 
In history, the amount by which male students of male instructors do worse on the final
is weakly significant ($p$-value 0.10):
male history students give significantly higher SET scores to male instructors, despite the fact that 
they appear to learn more from female instructors. 
SET do not appear to measure teaching effectiveness, at least not primarily.

\begin{table}[htbp]
  \centering
  \footnotesize 
  \caption{Association between student performance and gender concordance}
    \begin{tabular}{lccccc}
    \toprule 
          & \multicolumn{2}{c}{Male student}  &  & \multicolumn{2}{c}{Female student} \\
      & $\rho$  &  $p$-value &  & $\rho$  &  $p$-value    \\
                             \midrule
      \quad  Overall &                 -0.01       & 0.76 & &  0.01       & 0.65  \\
      \quad  History &                 -0.11       & 0.10 & &  0.01       & 0.86   \\
      \quad  Macroeconomics &           0.02       & 0.76 & & -0.00       & 0.97   \\
      \quad  Microeconomics &          -0.04       & 0.60 & &  0.00       & 0.94  \\
      \quad  Political sciences &       0.10       & 0.25 & &  0.03       & 0.76  \\
      \quad  Sociology &                0.02       & 0.85 & & -0.01       & 0.94  \\
    \bottomrule
    \end{tabular}%
 \label{tab:finalconcordance}%
  
  \textit{Note: $p$-values are two-sided.}
\end{table}%
\normalsize

\subsection{SET and grade expectations}
The next null hypothesis we test is that the pairing by course of average SET scores with
average interim grades is as if at random.
Because interim grades may set student grade expectations, we would expect that if
students give higher SET in courses where they expect higher grades, the association
would be positive.
Indeed, the association is positive and generally highly statistically significant 
(Table~\ref{tab:instructor_gender}). 
Political institutions is the only topic for which the correlation between interim grades and 
SET scores is not significant ($p$-value 0.19). 
The $p$-values for all other courses are nearly 0. 
The correlation coefficients are especially high in history (0.32) and sociology (0.27). 
They are also high in macroeconomics (0.22), microeconomics (0.19) and political sciences (0.16).


\begin{table}[htbp]
  \centering
  \footnotesize 
  \caption{Association between SET and interim grades}
    \begin{tabular}{lcc}
    \toprule 
                          & $\rho$  & $p$-value  \\
   \midrule
    Overall &                 0.10       & 0.00   \\
    History &                 0.32       & 0.00   \\
    Political institutions &  0.06       & 0.19     \\
    Macroeconomics &          0.22       & 0.00    \\
    Microeconomics &          0.19       & 0.00     \\
    Political sciences &      0.16       & 0.03     \\
    Sociology &               0.27       & 0.00     \\
    \bottomrule
    \end{tabular}%
 \label{tab:instructor_gender}%
  
  \textit{Note: $p$-values are one-sided.}
\end{table}%
\normalsize

In summary, the association between SET and final exam grades is not statistically significant,
but the association between SET and grade expectations is positive and significant.
The association between instructor gender and SET is statistically significant---male instructors
get higher SET---but if anything, students of male instructors do worse on final exams
than students of female instructors.
Male students tend to give male instructors higher SET, even though they appear to learn less
than they do from female instructors.
We conclude that SET are influenced more by instructor gender and student grade expectations than
by teaching effectiveness.


\section{The US Randomized Experiment} \label{sec:US-results}

The previous section suggests that SET have little
connection to teaching effectiveness, but the natural experiment 
does not enable us to control for potential differences in teaching 
styles across instructors. 
\citet{MacNell2014} does.
As discussed above used SET from an online course in which 
43 students were randomly assigned to four\footnote{%
   As discussed above, there were six sections in all, of which two were taught by the professor and four
   were taught by TAs.
} 
discussion groups, each taught by one of two TAs, one male and one female. 
The TAs gave similar feedback to students, returned assignments at exactly the same time, etc., 
minimizing differences in teaching styles and grading.

Biases in student ratings are revealed by differences in ratings for a given TA when that 
TA is identified as male versus as female.
\citet{MacNell2014} find that ``the male identity received significantly higher scores on professionalism, promptness, fairness, respectfulness, enthusiasm, giving praise, and the
student ratings index \textellipsis Students in the two groups that perceived their assistant
instructor to be male rated their instructor significantly higher than did the students in the
two groups that perceived their assistant instructor to be female, regardless of the actual gender
of the assistant instructor.'' 
\citet{MacNell2014} used parametric tests whose assumptions did not match their experimental
design; part of our contribution is to show that their data admit a more rigorous analysis
using permutation tests that honor the underlying randomization and that avoid parametric
assumptions about SET. 
The new analysis supports their conclusions, in some cases even more strongly than the original
analysis.

We use permutation tests as described above in section~\ref{sec:methods}.
Individual $i$ is a student; the treatment is the combination of the TA's identity and the TA's
apparent gender (there are $K=4$ treatments).

The null hypothesis is that each student would give a TA the same SET score,
whether that TA is apparently male or apparently female.
A student might give the two TAs different scores,
and different students might give different scores to the same TA. 

Because of how the experimental randomization was performed, all allocations of students to 
TA sections that preserve the number of students in each section are equally likely,  
including allocations that keep the same students assigned to each actual TA constant.

To test whether there is a systematic difference in how students rate apparently male and 
apparently female TAs, we use the difference in pooled means as our test statistic:
We pool the SET for both instructors when they are identified as female
and take the mean, pool the SET for both instructors when they are identified as male
and take the mean, then subtract the second mean from the first mean
(Table~\ref{tab:macnell1}).
This is what \cite{MacNell2014} report as their main result.

As described above, the randomization is stratified and conditions on the set of students 
allocated to each of the two TAs, because, under the null hypothesis, we then know what 
SET students would have given for each possible allocation, completely specifying the
null distribution of the test statistic.
The randomization includes the nonresponders, who are omitted from the averages of the
group they end up in.

We also perform tests involving the association of concordance of student and apparent 
TA gender,
(Table~\ref{tab:macnell2}), SET and concordance of student and actual TA gender 
(Table~\ref{tab:macnell3}), and grades and actual TA gender (Table~\ref{tab:macnell4}).  
These tests use the sum of Pearson correlations across strata as the test statistic, 
but we find the $p$-value
based on the stratified permutation distribution of the test statistic, avoiding
parametric assumptions.

\subsection{SET and Perceived Instructor Gender}
\todo{make sure the category names make sense in the prose}
The first hypothesis we test is that students would rate a given TA the same,
whether the student thinks the TA is female or male. 
A positive value of the test statistic indicates that students give higher SET
to apparently male instructors.
There is weak evidence that the overall SET score depends on the perceived gender ($p$-value 0.12). 
The evidence is stronger for several other items students rated: fairness ($p$-value 0.01), 
promptness ($p$-value 0.01), giving praise ($p$-value 0.02), 
enthusiasm ($p$-value 0.06), communication ($p$-value 0.07), professionalism and respect 
(both criteria have $p$-values of 0.06). 
Items for which the $p$-values were greater than 0.10 include caring, clarity,
consistency, feedback, helpfulness, responsiveness, and knowledgeability.

\begin{table}[htbp]
  \centering
  \footnotesize 
  \caption{Mean ratings and reported instructor gender (male minus female)}
    \begin{tabular}{lcc}
    \toprule 
                          & difference in means  & $p$-value  \\
   \midrule
    Overall &                 0.47       & 0.12   \\
    Professional &            0.61       & 0.06   \\
    Respectful			   &  0.61       & 0.06   \\
    Caring &                  0.52       & 0.10    \\
    Enthusiastic   &          0.57       & 0.06     \\
    Communicate        &      0.57       & 0.07     \\
    Helpful   &               0.46       & 0.18     \\
    Feedback   &              0.47       & 0.16     \\
    Prompt    &               0.80       & 0.01     \\
    Consistent   &            0.46       & 0.21     \\
    Fair   &                  0.76       & 0.01     \\
    Responsive   &            0.22       & 0.48     \\
    Praise   &                0.67       & 0.02     \\
    Knowledge   &             0.35       & 0.29     \\
    Clear   &                 0.41       & 0.29     \\
    \bottomrule
    \end{tabular}%
 \label{tab:macnell1}%
  
  \textit{Note: $p$-values are two-sided.}
\end{table}%
\normalsize


We also conducted separate tests by student gender.
In contrast to our findings for the French data, where male students 
rated male instructors higher, 
for the \citet{MacNell2014} data perceived male instructors received 
significantly higher evaluation scores because female students rated the perceived 
male instructors higher (Table ~\ref{tab:macnell2}). 
Male students rated the perceived male instructor significantly (though weakly) 
higher on only one criterion: fairness ($p$-value 0.09). 
Female students, however, rated the perceived male instructor higher on overall satisfaction 
($p$-values of 0.11) and most teaching dimensions: 
praise ($p$-value 0.01), 
enthusiasm ($p$-value 0.05), 
caring ($p$-value 0.05), 
fairness ($p$-value 0.04), 
respectfulness ($p$-value 0.09),  
communication ($p$-values of 0.10), 
professionalism ($p$-value 0.09), 
and feedback ($p$-value 0.10). 
There is a negative but statistically insignificant correlation between (perceived) 
female instructors
and ratings on helpfulness, promptness, consistency, responsiveness, 
knowledge, and clarity.

\todo{For consistency, I think we should report differences in means rather than $\rho$.}
\begin{table}[htbp]
  \centering
  \footnotesize 
  \caption{SET and reported instructor gender, by gender concordance}
    \begin{tabular}{lccccc}
    \toprule 
          & \multicolumn{2}{c}{Both male}  &  & \multicolumn{2}{c}{Both female} \\
                          & $\rho$  &  $p$-value &  & $\rho$  & $p$-value    \\
                          
   \midrule
    Overall &                0.09       & 0.81 & & -0.36    & 0.11   \\
    Professional &           0.22       & 0.40 & & -0.36    & 0.09   \\
    Respectful			   & 0.22       & 0.33 & & -0.36    & 0.09   \\
    Caring &                 0.02       & 1.00 & & -0.46    & 0.05  \\
    Enthusiastic   &         0.09       & 0.82 & & -0.44    & 0.05   \\
    Communicate        &     0.12       & 0.67 & & -0.39    & 0.10  \\
    Helpful   &              0.21       & 0.42 & & -0.24    & 0.35   \\
    Feedback   &             0.04       & 0.90 & & -0.37    & 0.10   \\
    Prompt    &              0.38       & 0.15 & & -0.37    & 0.12   \\
    Consistent   &           0.07       & 0.84 & & -0.34    & 0.17   \\
    Fair   &                 0.41       & 0.09 & & -0.43    & 0.04  \\
    Responsive   &           0.18       & 0.53 & & -0.03    & 1.00  \\
    Praise    &              0.29       & 0.26 & & -0.47    & 0.01  \\
    Knowledge   &            0.08       & 0.77 & & -0.29    & 0.21  \\
    Clear   &                0.06       & 0.76 & & -0.25    & 0.29  \\
    \bottomrule
    \end{tabular}%
 \label{tab:macnell2}%

  \textit{Note: $p$-values are two-sided.}
\end{table}%
\normalsize


\todo{For consistency, I think we should report differences in means rather than $\rho$.}
\begin{table}[htbp]
  \centering
  \footnotesize 
  \caption{SET and actual instructor gender, by gender concordance}
    \begin{tabular}{lccccc}
    \toprule 
          & \multicolumn{2}{c}{Both male}  &  & \multicolumn{2}{c}{Both female} \\
                          & $\rho$  &  $p$-value &  & $\rho$  &  $p$-value    \\
                          
   \midrule
    Overall &                -0.07       & 0.72 & &  0.13    & 0.57   \\
    Professional &            0.08       & 0.75 & &  0.04    & 0.95   \\
    Respectful	      &  0.08       & 0.83 & &  0.04    & 0.94   \\
    Caring &                 -0.11       & 0.59 & &  0.03    & 0.98  \\
    Enthusiastic   &         -0.07       & 0.82 & &  0.20    & 0.39   \\
    Communicate        &     -0.01       & 0.84 & &  0.08    & 0.67  \\
    Helpful   &               0.01       & 0.96 & & -0.12    & 0.70   \\
    Feedback   &             -0.12       & 0.69 & &  0.17    & 0.51   \\
    Prompt    &              -0.05       & 0.88 & &  0.14    & 0.52   \\
    Consistent   &            0.05       & 0.85 & &  0.17    & 0.48   \\
    Fair   &                 -0.03       & 0.88 & &  0.28    & 0.23  \\
    Responsive   &           -0.06       & 0.84 & &  0.35    & 0.12  \\
    Praise   &                0.01       & 1.00 & &  0.34    & 0.13  \\
    Knowledge   &             0.11       & 0.70 & &  0.24    & 0.37  \\
    Clear   &                -0.12       & 0.64 & &  0.35    & 0.13  \\
    \bottomrule
    \end{tabular}%
 \label{tab:macnell3}%

  \textit{Note: $p$-values are two-sided.}  
\end{table}%
\normalsize

Students of both genders rated the apparently male instructor higher on all
dimensions, by an amount that often was statistically significant for female students 
(Table~\ref{tab:macnell2}).
However, students rated the actual male instructor higher on some dimensions
and lower on others, by amounts that never rose to statistical significance
(Table~\ref{tab:macnell3}). 

Students of the actual male instructor performed better in the course on average,
by an amount that was statistically significant (Table~\ref{tab:macnell4}). 
There is no statistical difference between student performance by 
perceived gender of the instructor. 

\begin{table}[htbp]
  \centering
  \footnotesize 
  \caption{Mean grade and instructor gender (male minus female)}
    \begin{tabular}{lcc}
    \toprule 
                     & difference in means   & $p$-value    \\
   \midrule
    Perceived &         1.75       & 0.54      \\
    Actual  &            -6.81       & 0.02      \\
    \bottomrule
    \end{tabular}%
 \label{tab:macnell4}%
 
\textit{Note: $p$-values are two-sided.}
\end{table}%
\normalsize

These results suggest that students rate instructors more on the basis of the 
instructor's perceived gender than on the basis of the instructor's effectiveness. 

\section{Code, Data, and Reproducibility}
IPython notebooks containing our analyses are at
\url{https://github.com/kellieotto/SET-and-Gender-Bias}.
The US data are available at \url{http://n2t.net/ark:/b6078/d1mw2k}.
The French data are not available, owing to French privacy law.

\section{Technical Notes}
We did not adjust the tests reported above for multiplicity.
We performed a total of approximately 50 tests on the French data, of which we
consider four to be our primary results:
\begin{enumerate}
    \item lack of association between SET and final exam scores (a negative result,
               so multiplicity is not an issue)
    \item lack of association between instructor gender and final exam scores (a negative result,
               so multiplicity is not an issue)
    \item association between SET and instructor gender
    \item association between SET and interim grades
\end{enumerate}        
Bonferroni's adjustment for these four tests would \todo{fix me when the results are in}

We performed a total of 77 tests on the US data.
We consider the three primary null hypotheses to be that 
(1)~perceived instructor gender plays no role in SET,
(2)~male students rate perceived male and female instructors the same, and
(3)~female students rate perceived male and female instructors the same.
To account for multiplicity, we tested these three ``omnibus'' hypotheses 
using the nonparametric combination of tests (NPC) method with Fisher's combining 
function~\citep[Chapter 4]{pesarinSalmaso10} to combine the 15 dimensions of teaching into
a single test statistic that measures how ``surprising'' the 15 observed differences would be
for each of the three null hypotheses.
In $10^5$ replications, the empirical $p$-values for the omnibus hypotheses were \todo{???},
giving upper 99\% confidence bounds of \todo{????} for the actual $p$-values (the upper bounds
were obtained by inverting Binomial hypothesis tests).
Thus, we reject hypotheses \todo{which?}.

We made no attempt to optimize the tests to have power 
against the alternatives considered.
For instance, with the US data, the test statistic
grouped the two identified-as-female sections and the two identified-as-male conditions,
in keeping with how \citet{MacNell2014} tabulated their results,
rather than using each TA as his or her own control (although the randomization keeps the 
two strata intact). 
Given the relatively small number of students in the US experiment, it is remarkable that
\emph{any} of the $p$-values is small, much less that the $p$-values for the omnibus
tests are effectively zero.

\section{Discussion and Conclusions}
To our knowledge, only two experiments have controlled for teaching style in
their designs: \citet{Arbuckle2003} and \citet{MacNell2014}. 
In both experiments, students generally give higher SET when they \emph{think} the instructor
is male, regardless of the actual gender of the instructor.
Both experiments found that systematic differences in SET by instructor gender reflect gender bias 
rather than a match of teaching style and student learning style or differences
in teaching effectiveness. 

\citet{Arbuckle2003} showed a group of 352 students 
``slides of an age- and gender-neutral stick figure and listened to a neutral voice 
presenting a lecture and then evaluated it on teacher evaluation forms that indicated 
1 of 4 different age and gender conditions 
(male, female, ``old,'' and ``young'')'' \citealp[p.507]{Arbuckle2003}. 
The experiment sought to measure whether 
``students' perceptions of a professor's age and gender influence their perceptions of the 
professor's warmth and enthusiasm.'' 
Students saw the same stick figure and heard the same voice, so differences in SET 
could be attributed to the age and gender the students were \emph{told} the instructor had.
\citet{Arbuckle2003} found that students rated apparently young male instructors higher 
than the other three combinations, especially on ``enthusiasm,'' ``showed interest in subject,'' 
and ``using a meaningful voice tone.'' 

We used permutation tests to examine data collected  by
\citet{Boring2015} and \citet{MacNell2014}, both of which find that gender biases prevent 
SET from measuring teaching effectiveness accurately and fairly. 
SET are more strongly related to instructor's perceived gender and to grade expectations 
than they are to learning, 
as measured by performance on anonymously graded, uniform final exams. 
The extent and direction of gender biases depend on context, so it is
impossible to adjust for such biases to level the playing field.
While the French university data show a positive male student bias for male instructors, 
the experimental US setting suggests a positive female student bias for male instructors.
And the biases in the French university data vary by course topic.
We would also expect the bias to depend on class size, format, level, physical characteristics
of the classroom, instructor ethnicity and a host of other variables, some of which
are discussed below.

We do not claim that there is \emph{no} connection between SET and student
performance.
However, the observed association is sometimes positive and sometimes
negative, and in general is not statistically significant---in contrast to
the statistically significant associations between SET and grade expectations and 
between SET and instructor gender.
SET appear to measure student satisfaction more than they measure teaching 
effectiveness \citep{StarkFreishtat2014}. 
While student satisfaction may \emph{contribute} to teaching effectiveness, it is not 
itself teaching effectiveness.
Students may be satisfied or dissatisfied with courses for reasons unrelated to 
learning outcomes---and not in the instructor's control (e.g., the instructor's gender).

Instructor race has also been shown to be associated with SET.
In the US, SET of instructors of color appear to be biased downwards:
minority instructors tend to receive significantly lower SET scores compared to white (male) 
instructors \citep{Merritt2008}.\footnote{%
  French law does not allow the use of race-related variables in data sets. 
  We were thus unable to test for racial biases in SET using the French data.
} 
Age, \citep{Arbuckle2003}, 
charisma \citep{Shevlin2000}, and 
physical attractiveness \citep{Riniolo2006,Hamermesh2005} 
are also associated with SET.
Other factors generally not in the instructor's control that may affect SET scores include
class time, class size, mathematical or technical content, and the 
physical classroom environment \citep{Hill2010}.

Hundreds of studies cast doubt on the validity of SET as a measure of teaching effectiveness 
(see \citet{Pounder2007} for a review and \citet{Galbraith2012,Carrell2010a} for exemplars). 
Other studies find that gender and SET are not significantly associated \citep{Bennett1982,Centra2000,Elmore1974} and that SET are valid and reliable measures of teaching effectiveness \citep{Benton2012,Centra1977}.\footnote{%
  Some authors who have claimed that SET are valid have a financial interest in 
  developing SET instruments and conducting SET.
} 
The contradictions among conclusions suggests that if SET are ever valid, they
are not valid in general: universities are not justified in assuming that SET are broadly valid at their
institution, valid in any particular department, or valid for any particular course. 
Given the many sources of bias in 
SET and the unpredictable magnitude of their effects, as a practical matter it is impossible to adjust
SET for biases to make them a valid, useful measure of teaching effectiveness. 

In the US, SET have two primary uses: 
instructional improvement and personnel 
decisions, including hiring, firing, and promoting instructors. 
We recommend caution in the first use, and discontinuing the second use, 
given the strong student biases that 
influence SET, even on ``objective'' items such as how promptly instructors return
assignments.\footnote{%
  In 2009, the French Ministry of Higher Education and Research upheld a 1997 
  decision of the French State Council that public universities can use SET only to help 
  tenured instructors improve their pedagogy, and that the administration may not use 
  SET in decisions that might affect  tenured instructors' careers (c.f. \citet{Boring2015ofce}). 
}

We find that in two very different universities and in a broad range of course topics, 
SET are biased against female instructors, and that
they measure gender biases and grade expectations more than they measure teaching 
effectiveness.
There is no evidence that gender bias is the exception, rather than the rule.
Hence, the onus should be on universities 
that rely on SET for employment decisions to provide convincing affirmative evidence
that such use does not have disparate impact on women,
under-represented minorities, or other protected groups.

\bibliographystyle{abbrvnat}
\bibliography{SETs}

\end{document}

